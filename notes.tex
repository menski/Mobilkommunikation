\documentclass[german,12pt,a4paper]{article}
\usepackage{fullpage}
\usepackage[ngerman]{babel}
\usepackage[utf8]{inputenc}
\usepackage{listings}
\usepackage{verbatim}
\usepackage{enumerate}
\usepackage{graphicx}
\usepackage{float}
\usepackage{wrapfig}
\usepackage{color}
\usepackage[usenames,dvipsnames]{xcolor}
\usepackage[font=small,format=plain,labelfont=bf,up,textfont=it,up]{caption}
\usepackage{subfig}

\pagestyle{plain}
\pagenumbering{arabic}
\frenchspacing

\newcommand{\comments}[1]{}
\renewcommand{\baselinestretch}{1.55}

%Redefine the first level
\renewcommand{\theenumi}{\textbf{\alph{enumi})}}
\renewcommand{\labelenumi}{\theenumi}

\begin{document}

\title{\textbf{Notes SMS}}
\author{Sebastian Menski (734272), Martin Ohmann (734801) \\ \texttt{\{menski,ohmann\}@uni-potsdam.de}}
\date{\today}

\maketitle

\section{SMS traditionell}
\subsection{SMS-Spezifikation}
\begin{itemize}
	\item Verwaltung durch Short Message Service Center (SMSC) des Providers
	\item SMSC kann SMS an mobile Endgeräte senden, maximale Payload 140 octets (8-bit Tupel)
		\begin{itemize}
			\item 7-bit encoding: 160 chars
			\item 8-bit encoding: 140 chars
			\item 16-bit encoding: 70 chars
		\end{itemize}
	\item SMS können auch dazu benutzt werden Binärdaten zu senden, spezielle Anwendungen auf 
		dem Telefon übernehmen die Verarbeitung dieser, Beispiele:
		\begin{itemize}
			\item Klingelton-Download
			\item Austausch von Bildnachrichten
			\item Change look and feel of the handset's GUI
		\end{itemize}		
	\item System kann überlange Nachrichten segmentieren; mehrere SMS werden gesendet, Teil der Payload jeder
		dieser SMS wird für nutzerspezifischen Header für Sequenzinformationen verwendet
	\item SMSC arbeiten entweder nach store-and-forward oder forward-and-forget Prinzip:
		\begin{itemize}
			\item store-and-forward: System versucht Nachricht für eine festgelegte 
				Vorhaltezeit (meist 48h bis 7 Tage) nachzusenden, bis diese erfolgreich beim Empfänger 
				eingetroffen ist. Nachrichten, die innerhalb der Vorhaltezeit nicht zugestellt werden 
				konnten, werden vom SMSC gelöscht.
			\item forward-and-forget: Nachricht wird direkt an den Empfänger weitergeleitet, ohne Prüfung, ob 
				dieser die Nachricht erhalten hat und ohne Nachsenden im Fehlerfall
		\end{itemize}	
	\item SMS Protocol Stack besteht aus 4 Schichten:
		\begin{itemize}
			\item application layer
			\item transfer layer
			\item relay layer
			\item link layer
		\end{itemize}		
	\item \textit{Beispiel einer Transfer protocol data unit (SMS-Aufbau) anführen (Computer-SMS\_pdf.pdf, Table 1)}
\end{itemize}

\subsection{Shortcodes}
\begin{itemize}
	\item Abgekürzte Nummer (4-6 Ziffern) als ``Adresse'' für SMS
	\item Nur im Netz des Providers gültig, oder in den Netzen mehrerer Provider gültig (allgemeine shortcodes)
	\item Können dazu genutzt werden Nachrichten von oder an Nutzer in mehreren Mobilfunknetzen zu senden.
	\item Provider senden oft Informations-SMS unter Verwendung von nur im eigenen Netzwerk gültigen Shortcodes
	\item Shortcodes können angemietet werden und dem Nutzer kann es erlaubt sein, Keywords an diese zu senden, um 
		damit eine Aktion auszulösen (z.B. Klingelton-Download)
	\item Shortcodes können vom Mobilfunkvertrag abweichende Kosten haben, diese werden in der Telefonabrechnung 
		erfasst.
\end{itemize}

\subsection{Email-Gateway}
\begin{itemize}
	\item manche Provider bieten Email-to-SMS gateways an
	\item Email an 1234567890@provider.de löst SMS an 1234567890 aus
\end{itemize}

\subsection{SMS centers (SMSC)}
\begin{itemize}
	\item SMS-Nachrichten werden vo Telefon über Funk via Funkturm der Mobilfunkzelle an das SMSC des Providers 
		gesendet
	\item Die Zugriffsprotokolle des SMSC erlauben Interaktion zwischen zwei SMSCs oder zwischen Externen Shortmessage 
		Entities (SMEs) und einem SMSC.
	\item SMEs sind sind Software Anwendungen auf Netzwerkkomponenten (z.B. mobile handsets) oder Hardware-Geräte, welche 
		SMS senden und empfangen können
	\item Short Message Peer-to-Peer (SMPP) Protocol erlaubt Interaktion zwischen externen SMEs und SMSCs 
		verschiedener Hersteller.
	\item \textit{Grafik SMSC, SMPP, SMS Broker anführen (Computer-SMS\_pdf.pdf, Figure 1)}
\end{itemize}

\subsection{Message aggregators}
\begin{itemize}
	\item Provider kommunizieren in der Regel nicht direkt mit SMSCs, sondern nutzen einen SMS Broker (oder message aggregator)
	\item Ein Aggregator ist eine Wirtschaftseinheit, welcher Verträge mit Netzwerk-Providern aushandelt und als Mittelsmann
		Zugang zu Mobilfunknetzen zur Nachrichtenübertragung für Dritte ohne direkte Beziehung zum Mobilfunknetzwerk gewährt. 
	\item Der Aggregator nutzt SMPP um Verbindungen mit Provider-Mobilfunknetzwerken aufrechtzuerhalten.
	\item Aggregatoren bieten Zugriff auf ihre Server typischerweise über SMPP oder kundenspezifische APIs an.
	\item \textit{Grafik SMSC, SMPP, SMS Broker anführen (Computer-SMS\_pdf.pdf, Figure 1)}
\end{itemize}

\subsection{SMS-Versand in Deutschland, Statistiken}
\begin{itemize}
	\item SMS mittlerweile fast 20 Jahre alt (erste SMS überhaupt wurde am 3. Dezember 1992 verschickt)
	\item \textit{1999 - 2007, Grafik anfügen http://www.bitkom.org/de/presse/49919\_49417.aspx}
	\item \textit{2006 - 2011, Grafik anfügen http://www.bitkom.org/de/presse/64046\_67951.aspx}
	\item 2010: 1300 SMS pro Sekunde (78k/min)
	\item Stand 2011: 83\% aller Deutschen ab 14 Jahren besitzen ein Mobiltelefon: ca. 59Mio.
\end{itemize}

\subsection{SMS-Versand über GSM/GPRS}
\begin{itemize}
	\item 
\end{itemize}

\end{document}
