\documentclass[german,12pt,a4paper]{article}
\usepackage{fullpage}
\usepackage[ngerman]{babel}
\usepackage[utf8]{inputenc}
\usepackage{listings}
\usepackage{verbatim}
\usepackage{enumerate}
\usepackage{graphicx}
\usepackage{float}
\usepackage{wrapfig}
\usepackage{color}
\usepackage[usenames,dvipsnames]{xcolor}
\usepackage[font=small,format=plain,labelfont=bf,up,textfont=it,up]{caption}
\usepackage{subfig}

\pagestyle{plain}
\pagenumbering{arabic}
\frenchspacing

\newcommand{\comments}[1]{}
\renewcommand{\baselinestretch}{1.55}

%Redefine the first level
\renewcommand{\theenumi}{\textbf{\alph{enumi})}}
\renewcommand{\labelenumi}{\theenumi}

\begin{document}

\title{\textbf{Notes SMS}}
\author{Sebastian Menski (734272), Martin Ohmann (734801) \\ \texttt{\{menski,ohmann\}@uni-potsdam.de}}
\date{\today}

\maketitle

\section{SMS traditionell}
\subsection{SMS-Spezifikation}
\begin{itemize}
	\item Verwaltung durch Short Message Service Center (SMSC) des Providers
	\item SMSC kann SMS an mobile Endgeräte senden, maximale Payload 140 octets (8-bit Tupel)
		\begin{itemize}
			\item 7-bit encoding: 160 chars
			\item 8-bit encoding: 140 chars
			\item 16-bit encoding: 70 chars
		\end{itemize}
	\item SMS können auch dazu benutzt werden Binärdaten zu senden, spezielle Anwendungen auf 
		dem Telefon übernehmen die Verarbeitung dieser, Beispiele:
		\begin{itemize}
			\item Klingelton-Download
			\item Austausch von Bildnachrichten
			\item Change look and feel of the handset's GUI
		\end{itemize}		
	\item System kann überlange Nachrichten segmentieren; mehrere SMS werden gesendet, Teil der Payload jeder
		dieser SMS wird für nutzerspezifischen Header für Sequenzinformationen verwendet
	\item SMSC arbeiten entweder nach store-and-forward oder forward-and-forget Prinzip:
		\begin{itemize}
			\item store-and-forward: System versucht Nachricht regelmäßig für eine festgelegte 
				Vorhaltezeit (meist 48h bis 7 Tage) nachzusenden, bis diese erfolgreich beim Empfänger 
				eingetroffen ist. Nachrichten, die innerhalb der Vorhaltezeit nicht zugestellt werden 
				konnten, werden vom SMSC gelöscht. Ist Empfängernummer dem SMSC unbekannt, lehnt es diese sofort ab.
			\item forward-and-forget: Nachricht wird direkt an den Empfänger weitergeleitet, ohne Prüfung, ob 
				dieser die Nachricht erhalten hat und ohne Nachsenden im Fehlerfall
		\end{itemize}	
	\item SMS Protocol Stack besteht aus 4 Schichten:
		\begin{itemize}
			\item application layer
			\item transfer layer
			\item relay layer
			\item link layer
		\end{itemize}		
	\item \textit{Beispiel einer Transfer protocol data unit (SMS-Aufbau) anführen \\(Computer-SMS\_pdf.pdf, Table 1)}
\end{itemize}

\subsection{Übertragung}
\begin{itemize}
	\item SMS nutzt einen Signalisierungskanel des GSM-Standards wie etwa SDCCH (Stand-alone Dedicated Control Channel) 
		oder FACCH (Fast Associated Control Channel). Diese Kanäle werden auch genutzt, um Gespräche aufzubauen und zu 
		halten.
	\item Kurzmitteilungen können parallel zu Telefongesprächen empfangen/gesendet werden. Teil der Bandbreite des 
		Verkehrsdatenkanals wird hier temporär zum Signailiserungskanal (SACCH) umkonfiguriert.
\end{itemize}

\subsection{Nachrichtentypen}
\begin{itemize}
	\item Flash Message: erscheint direkt auf Display; die meisten Mobiltelefone können solche Nachrichten nicht speichern,
		Beispiel: Anzeige des Restguthabens direkt nach einem Gespräch.
	\item Silent Message: erscheinen weder auf dem Display noch durch akkustisches Signal. 
		Wird z.B. von Polizei zur Personenortung genutzt.
	\item Concatenated Message: Mehrere Kurzmitteilungen werden zu einer Nachricht zusammengestellt.
\end{itemize}

\subsection{Shortcodes}
\begin{itemize}
	\item Abgekürzte Nummer (4-6 Ziffern) als ``Adresse'' für SMS
	\item Nur im Netz des Providers gültig, oder in den Netzen mehrerer Provider gültig (allgemeine shortcodes)
	\item Können dazu genutzt werden Nachrichten von oder an Nutzer in mehreren Mobilfunknetzen zu senden.
	\item Provider senden oft Informations-SMS unter Verwendung von nur im eigenen Netzwerk gültigen Shortcodes
	\item Shortcodes können angemietet werden und dem Nutzer kann es erlaubt sein, Keywords an diese zu senden, um 
		damit eine Aktion auszulösen (z.B. Klingelton-Download)
	\item Shortcodes können vom Mobilfunkvertrag abweichende Kosten haben, diese werden in der Telefonabrechnung 
		erfasst.
\end{itemize}

\subsection{Email-Gateway}
\begin{itemize}
	\item manche Provider bieten Email-to-SMS gateways an
	\item Email an 1234567890@provider.de löst SMS an 1234567890 aus
\end{itemize}

\subsection{SMS centers (SMSC)}
\begin{itemize}
	\item SMS-Nachrichten werden vo Telefon über Funk via Funkturm der Mobilfunkzelle an das SMSC des Providers 
		gesendet
	\item Die Zugriffsprotokolle des SMSC erlauben Interaktion zwischen zwei SMSCs oder zwischen Externen Shortmessage 
		Entities (SMEs) und einem SMSC.
	\item SMEs sind sind Software Anwendungen auf Netzwerkkomponenten (z.B. mobile handsets) oder Hardware-Geräte, welche 
		SMS senden und empfangen können
	\item Short Message Peer-to-Peer (SMPP) Protocol erlaubt Interaktion zwischen externen SMEs und SMSCs 
		verschiedener Hersteller.
	\item \textit{Grafik SMSC, SMPP, SMS Broker anführen \\(Computer-SMS\_pdf.pdf, Figure 1)}
\end{itemize}

\subsection{Message aggregators}
\begin{itemize}
	\item Provider kommunizieren in der Regel nicht direkt mit SMSCs, sondern nutzen einen SMS Broker (oder message aggregator)
	\item Ein Aggregator ist eine Wirtschaftseinheit, welcher Verträge mit Netzwerk-Providern aushandelt und als Mittelsmann
		Zugang zu Mobilfunknetzen zur Nachrichtenübertragung für Dritte ohne direkte Beziehung zum Mobilfunknetzwerk gewährt. 
	\item Der Aggregator nutzt SMPP um Verbindungen mit Provider-Mobilfunknetzwerken aufrechtzuerhalten.
	\item Aggregatoren bieten Zugriff auf ihre Server typischerweise über SMPP oder kundenspezifische APIs an.
	\item \textit{Grafik SMSC, SMPP, SMS Broker anführen \\(Computer-SMS\_pdf.pdf, Figure 1)}
\end{itemize}

\subsection{SMS Technische Realisierung über GSM}
\begin{itemize}
	\item Realisiert durch den Mobile Application Part (MAP) des SS\#7 Protokolls (Signaling System No. 7)
	\item Elemente des Short Message Protokolls werden als Felder innerhalb von MAP Nachrichten 
		durch das Netzwerk transportiert
	\item Vier MAP Prozeduren zur Kontrolle des Short Message Services:
		\begin{itemize}
			\item Mobile Originated (MO) short message service transfer
			\item Mobile Terminated (MT) short message service transfer
			\item Short message alert procedure
			\item Short message waiting data set procedure
		\end{itemize}
\end{itemize}

\subsubsection{MO short message service transfer}
\begin{itemize}
	\item Immer wenn ein Nutzer eine SMS versendet, wir diese zunächst zum 
		VMSC/SGSN (Mobile Switching Center/GRPS Core Network) geschickt
	\item Neben der Textnachricht werden Zieladresse und die Adresse des 
		SMSCs übermittelt (Addresse des SMSCs in der Telefonkonfiguration der SIM-Karte 
		definiert)
	\item VMSC/SGSN ruft MAP service package MAP\_MO\_FORWARD\_SHORT\_MESSAGE auf um 
		Nachricht an das spezifizierte Interworking MSC des Service Centers zu senden
	\item Der Service sendet die mo-ForwardSM MAP Operation an das SMSC, welches 
		in der Nachricht angegeben wurde. Eingebettet in Transaction Capabilities Application
		Part (TCAP). Tranport zum Core Network erfolgt über den Signaling Ceonnection 
		Control Prt (SCCP)
	\item Nach Erhalt der MAP mo-ForwardSM Nachricht gibt das Interworking MSC des SMSC die 
		SMS-PP Application Protocal Data Unit (APDU) (beinhaltet den Nachrichtentext) an 
		das zuständige Service Center des SMSC weiter. Hier wird sie gespeichert und 
		anschließend versucht an die Zieladresse auszuliefern. SC gibt Bestätigung von 
		Fehler der Erfolg zurück.
	\item Nach Erhalt des Submission Satus vom SC sendet das Interworking MSC eine 
		Benachrichtung zurück an das VMSC/SGSN des Senders, von dort aus wird die 
		Benachrichtigung über Funk an den Sender ausgeliefert
	\item \textit{Grafik MO anfügen \\(http://en.wikipedia.org/wiki/Short\_message\_service\_technical\_realisation\_(GSM))}
\end{itemize}

\subsubsection{MT short message service transfer}
\begin{itemize}
	\item Wenn SMSC feststellt, dass es eine SMS ausliefern soll, dann sendet es die 
		SMS-PP APDU der an die Gateway MSC (GMSC) Kompomente des SMSC.
	\item Das GMSC muss nach dem Empfangen der Nachricht die Position des Empfängers 
		herausfinden (in diesem Zusammenhang ist der Begriff Gateway MSC ein MSC das 
		Routing Informationen aus dem Home Location Register (HLR) bezieht). 
		Dies geschieht, indem das GMSC das MAP service package \\
		MAP\_SEND\_ROUTING\_INFO\_FOR\_SM aufruft, welches eine \\ sendRoutingInfoForSM 
		(SRI-for-SM) MAP Nachricht an das HLR der Zielnummer sendet, um die 
		aktuelle Position des Empfängers anzufragen
	\item Die SRI-for-SM-Nachricht kann entweder an ein HLR im gleichen Netzwerk, oder 
		aber über eine Verbindung zu einem HLR in einem fremden PLMN (Public land mobile 
		network), gesendet werden (je nachdem zu welchem Netzwerk der Empfänger gehört)
	\item Das HLR macht einen Database lookup und sendet Daten zur aktuellen Position des 
		Empfängers an das GMSC des SMSCs zurück. Diese Positionsdaten können entweder die 
		Adresse des MSC sein, über welches der Empfänger gerade erreichbar ist, die 
		SGSN Adresse, oder beides. Das HLR kann ebenfalls eine Fehlermeldung zurücksenden, 
		falls der Empfänger nicht für Kurznachrichten geeignet ist.
	\item Sobald das GMSC die Routing Informationen erhalten hat, versucht des die 
		Nachricht an den Empfänger zu übermitteln. Dies geschieht durch Aufrufen des \\
		MAP\_MT\_FORWARD\_SHORT\_MESSAGE services, welcher eine \\ 
		MAP mt-ForwardSM Nachricht 
		an die vom HLR bereitgestellte Adresse sendet. Dabei ist es egal, ob es 
		sich hierbei um ein MSC oder SGSN handelt
	\item Das VMSC fragt die benötigten Informationen zum Ausliefern der Nachricht beim VLR 
		durch eine Send\_Info\_for\_MS\_SMS Nachricht ab. Das VLR sucht daraufhin nach dem 
		Empfänger und liefert die Mobile Subscriber ISDN Number (MSISDN) des Empfängers an 
		das VMSC zurück (im Fehlerfall wird Fehlermeldung gesendet; in diesem Fall wird 
		Auslieferung der SMS abgebrochen und beim SMSC gemeldet. Im Erfolgsfall sendet das VMSC Erflogsnachricht an SMSC)
	\item GMSC Komponente des SMSC reicht das Ergebnis des Zustellungsversuchs an das SC 
		weiter. Im Erfolgsfall wird die Nachricht aus der Store and Forward Engine (SFE) 
		entfernt und, falls angefordert, ein Zustellreport an den Sender geschickt.
	\item Bei einem gescheiterten Zustellungsveruch versucht das SMSC periodisch die 
		Nachricht für eine bestimmte Vorhaltezeit zuzustellen; SMSC registiert sich 
		eventuell beim HLR, um über die Verfügbarkeit des Empfängers benachrichtigt zu 
		werden.
	\item \textit{Grafik MT anfügen \\(http://en.wikipedia.org/wiki/Short\_message\_service\_technical\_realisation\_(GSM))}
\end{itemize}

\subsubsection{Failed short message delivery}
\begin{itemize}
	\item Text
\end{itemize}

\subsubsection{MAP Transport Protocols}
\begin{itemize}
	\item Text
\end{itemize}

\subsection{SMS-Versand in Deutschland, Statistiken}
\begin{itemize}
	\item SMS mittlerweile fast 20 Jahre alt (erste SMS überhaupt wurde am 3. Dezember 1992 verschickt)
	\item \textit{1999 - 2007, Grafik anfügen http://www.bitkom.org/de/presse/49919\_49417.aspx}
	\item \textit{2006 - 2011, Grafik anfügen http://www.bitkom.org/de/presse/64046\_67951.aspx}
	\item 2000: 11,4 Mrd. SMS
	\item 2005: \textgreater 22 Mrd. SMS
	\item 2010: \textgreater 41 Mrd. SMS, 1300 SMS pro Sekunde (78k/min)
	\item Stand 2011: 83\% aller Deutschen ab 14 Jahren besitzen ein Mobiltelefon: ca. 59Mio.
\end{itemize}

\subsection{MMS}
\begin{itemize}
	\item nutzt Multimedia Messaging Service Center (MMSC)
	\item Übertragung über GPRS, Kommunikation mit Endgerät über WAP
	\item Spezifikation beinhaltet Schnittstellen zur Kommunikation mit E-Mail-Gateways und anderen MMSCs, die auf 
		SMTP beruhen; zur Komm. mit Value Added Services (VAS) wird SOAP verwendet
	\item Kodierung des Nachrichtenbodies basiert auf MIME
	\item MMS im Vergleich zu SMS viel stärker an E-Mail angelehnt
\end{itemize}

\end{document}
