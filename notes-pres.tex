\documentclass[german,11pt,a4paper]{article}
\usepackage{fullpage}
\usepackage[ngerman]{babel}
\usepackage[utf8]{inputenc}
\usepackage{listings}
\usepackage{verbatim}
\usepackage{enumerate}
\usepackage{graphicx}
\usepackage{float}
\usepackage{wrapfig}
\usepackage{color}
\usepackage[usenames,dvipsnames]{xcolor}
\usepackage[font=small,format=plain,labelfont=bf,up,textfont=it,up]{caption}
\usepackage{subfig}

\pagestyle{plain}
\pagenumbering{arabic}
\frenchspacing

\newcommand{\comments}[1]{}
\renewcommand{\baselinestretch}{1.2}

%Redefine the first level
\renewcommand{\theenumi}{\textbf{\alph{enumi})}}
\renewcommand{\labelenumi}{\theenumi}

\begin{document}

\title{\textbf{Notes SMS über GSM/GPRS}}

\maketitle

\section{Entstehung}
\begin{itemize}
	\item Ursprünglicher Konzeptvorschlag 1984 von Friedhelm Hillebrand von der 
		damaligen Deutschen Bundespost
\end{itemize}

\section{SMS-Spezifikationen}
\begin{itemize}
	\item andere Encodings: 140 8-bit-Zeichen, 70 16-bit-Zeichen
	\item SMS können auch für Binärdaten verwendet werden, z.B. Klingeltöne
	\item Bei Segmentierung: Teil der Payload wird für nutzerspezifischen Header mit 
		Sequenzinformationen verwendet
	\item Paralleles Senden/Empfangen während Telefonat möglich; Bandbreite des 
		Signailiserungskanals wird dann aufgeteilt
	\item SDCCH: Stand-alone Dedicated Control Channel
	\item FACCH: Fast Associated Control Channel
	\item Store-and-forward: SMSC speichert Nachricht und versucht diese periodisch 
		für eine festgelegte Vorhaltezeit (meist 48h bis 7 Tage) zuzustellen. 
		SMSC löscht Nachricht nach erfolgreicher Zustellung oder nach Ablauf der 
		Vorhaltezeit
\end{itemize}

\section{SMS-Aufbau}
\begin{itemize}
	\item Header:
		\begin{itemize}
			\item 1 Byte -- Länge der SMSC-Info
			\item 1 Byte -- Typ der SMSC-Adresse
			\item n Byte -- SMSC-Adresse
			\item 1 Byte -- erstes Byte der SMS-DELIVER Message
			\item 1 Byte -- Länge Sendernummer 
			\item n Byte -- Sendernummer
			\item 1 Byte -- Protocol Identifier
			\item 1 Byte -- Datenkodierunge
			\item 7 Byte -- Timestamp
			\item 1 Byte -- Länge Payload
		\end{itemize}
	\item Body:
		\begin{itemize}
			\item n Byte -- Payload
		\end{itemize}	
\end{itemize}

\section{SMS-Versand}
\begin{itemize}
	\item 
\end{itemize}

\section{Grenzen/Nachteile}
\begin{itemize}
	\item Multi-Device:
		\begin{itemize}
			\item User nutzen Laptops, Tablets, Smartphones und PCs zur Kommunikation
			\item Möglichkeit der Nutzung eines Accouns auf verschiedenen Systemen ist hier 
				essentiell
		\end{itemize}
	\item Kosten:
		\begin{itemize}
			\item Rechenbeispiel SMS: 20 Cent/SMS -- bei 140 Byte/SMS -- 0.14 Cent/Byte
			\item Rechenbeispiel Daten: \$30/2GB-Flat -- 0.000003 Cent/Byte
		\end{itemize}
	\item Real-Time:
		\begin{itemize}
			\item Apples iMessage z.B. bietet ``is typing''-Feature und kommt mit 
				geringeren Zustellzeiten aus
		\end{itemize}
	\item Gruppen:
		\begin{itemize}
			\item senden an mehrere Personen zwar möglich, jedoch keine Möglichkeit 
				für Empfänger allen Personen zu antworten, da immer nur der Absender 
				übermittelt wird 
		\end{itemize}
	\item Rich-Text:
		\begin{itemize}
			\item Eingebettete Bilder und Videos per MMS zwar möglich, aber teuer
		\end{itemize}
	\item Übertragungskanal:
		\begin{itemize}
			\item 2006: 20 Mrd. SMS in Dtl., 2011 46Mrd.
		\end{itemize}
\end{itemize}

\end{document}
